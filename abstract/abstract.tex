\documentclass[11pt, letterpaper]{article}
\usepackage[utf8]{inputenc}
\usepackage{caption} % for table captions
\usepackage{amsmath} % for multi - line equations and piecewises
\usepackage{graphicx}
%\usepackage{textcomp}
\usepackage{xspace}
\usepackage{verbatim} % for block comments
%\usepackage{subfig} % for subfigures
\usepackage{enumitem} % for a) b) c) lists
\newcommand{\Cyclus}{\textsc{Cyclus}\xspace} %
\newcommand{\Cycamore}{\textsc{Cycamore}\xspace} %
\usepackage{tabularx}
\usepackage{color}
\usepackage{setspace}
\definecolor{bg}{rgb}{0.95, 0.95, 0.95}
\newcolumntype{b}{X}
\newcolumntype{f}{ > {\hsize=.15\hsize}X}
\newcolumntype{s}{ > {\hsize=.5\hsize}X}
\newcolumntype{m}{ > {\hsize=.75\hsize}X}
\newcolumntype{r}{ > {\hsize=1.1\hsize}X}
\usepackage{titling}
\usepackage[hang, flushmargin]{footmisc}
\renewcommand *\footnoterule{}
\graphicspath{{images /}}

\title{Developing back-end capabilities in Cyclus, a fuel cycle simulator
        \\ \vspace{0.5em} TWOFCS2019 Abstract}
\author{Gwendolyn J. Chee}


\begin{document}
	\maketitle
	\hrule

\section * {}
\doublespacing
Implementation of a nuclear waste disposal plan and minimizing the cost of the 
nuclear fuel cycle are crucial to the future use of nuclear power 
\cite{massachusetts_institute_of_technology_future_2003}. 
If the U.S. nuclear industry does not find an effective and safe plan to manage 
the waste, the nuclear industry will continue facing political and social 
opposition. 
It has been shown that permanent underground disposal of nuclear waste provides
excellent isolation from the human-inhabited environment 
\cite{rechard_evolution_2014}. 
Therefore, this work relies on the expectation that the chosen method of long 
term disposal of spent nuclear fuel (SNF) will be a deep geologic repository. 

Previous work towards the wicked problem of getting spent nuclear fuel from reactor 
sites to a final waste repository focuses on how different waste acceptance strategies 
impact economic expenditure \cite{nesbit_proposed_2015}, pre-emplacement 
surface storage time, waste package size, and repository 
footprint \cite{greenberg_application_2012}. 
Previous work in studying repository loading have used spent fuel assemblies 
that have an average burn up composition \cite{johnson_optimizing_2016} 
to evaluate the heat load in the repository \cite{greenberg_application_2012}. 

Therefore, instead of using average SNF composition, this work aims to use U.S. 
historical SNF inventory data \cite{peterson_unf_standards_2017} in various 
simulations to study how waste acceptance strategies impacts the heat distribution
in the repository. 
This research aims to contribute realistic modeling tools for simulating how 
waste acceptance strategies impact the heat distribution in the repository 
since most of the current research is focused on the economic and 
transportation components of the problem. 

These simulations will be performed using \textsc{Cyclus}, an 
\textit{agent-based} fuel cycle simulation framework. 
In \textsc{Cyclus}, each facility in the fuel cycle is modeled individually 
and they interact with one another as independent \textit{agents}. 
The goal of this work is to develop a waste conditioning \textit{agent}, 
an interim storage \textit{agent} and a waste repository \textit{agent} 
to provide \textsc{Cyclus} with the capabilities to run these simulations. 
Previously, a repository agent was created \cite{huff_cyclus_2013}, however, 
it is no longer compatible with the current \textsc{Cyclus} and 
did not have the capabilities to run these proposed simulations.  

The conditioning facility packages spent fuel assemblies into a waste canister 
that has user defined properties such as radius length and material thermal 
conductivity. 
The repository facility emplaces waste canisters in the repository in a
systematic order. 
It only accepts canisters that results in the repository 
remaining below the thermal limit of the host geologic media. 
It provides the user with a spatial and time dependent temperature 
distribution in the repository. 
The interim storage facility is the brains of the operation and gives 
waste canisters to the repository facility to emplace based on a waste 
acceptance strategy. 
First-in-first-out and last-in-first-out fuel allocation strategies 
are considered. 


\bibliographystyle{unsrt}
\bibliography{bibliography.bib}

\end{document}
